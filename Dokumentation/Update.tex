\subsection{Updatemechanismus - Funktionsweise (DH)}
Um eine Update-Funktion zu implementieren, wird ein eigenes Programm (eine eigene Exe) erstellt. Zu beachten ist, dass dieses Programm selber nicht aktualisiert werden kann. Der Grund hierfür liegt in der Aufgabe des eigenständigen Programms: Es überprüft auf einem Server, ob neue oder geänderte Dateien vorliegen, lädt diese herunter und installiert sie. Die heruntergeladenen Dateien können DLLs, Konfiguration oder ein Programm (Exe) sein. Das Programm, welches diese Dateien herunterlädt, kann vom Hauptprogramm aus ausgeführt werden. Dann muss das Hauptprogramm aber nach Ausführung des Update-Programms beendet und nach dem Update-Prozess wieder neu gestartet werden \cite[vgl.][]{HOW_Update}.\\
Was geschieht bei der Installation? Kurz gesagt, werden alle notwendigen Dateien an die richtige Stelle kopiert, so dass sie vom Hauptprogramm aus gefunden werden. Außerdem können Umgebungsvariablen und Registry-Einträge geschrieben sowie Verknüpfungen erstellt werden. Es geschehen jedoch noch weitere Vorgänge während einer Installtion, der Installationsprozess wird im Kapitel zur Erstellung des Installers genauer erläutert.\\
Die Hauptanwendung kann vorerst ohne Rücksicht auf den Update-Prozess entwickelt werden, wodurch die Funktion des Updates von der Priorität nach ganz hinten rückt. Zu beachten ist lediglich, dass sich die Datenbank, also die Tabellendefinitionen, nicht ändern dürfen. Eine Aktualisierung der Datenbank ist sehr aufwendig und fehleranfällig. Aus diesem Grund wird strikt darauf geachtet, dass die Datenbank nicht mehr verändert werden muss.
\newpage