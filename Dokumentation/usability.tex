\subsection{Usabilityaspekte (FA)}
\label{USABILITY_ASPEKTE}
In folgendem sollen einige wichtige Informationen zur Usability bekannt gegeben werden. Dabei wird zuerst ein Überblick aufgezeigt, welcher erklärt, was für Aspekte zu dem Thema Usability gehören und für dieses Projekt relevant sind. Daraufhin werden Normen präsentiert, welche die Themen Gebrauchstauglichkeit und Dialoggestaltung detailliert behandeln. Auf die dort geschilderten Aspekte und Grundsätze ist innerhalb der Projektarbeit Rücksicht zu nehmen.

\subsubsection{Usability-Kennzahlen}
Um Usability zu definieren, ist es wichtig aktuelle Qualitätskriterien für Software zu kennen. Eines davon ist der Human Factor, dessen wichtigstes Kriterium die Benutzerfreundlichkeit ist. Unter Benutzerfreundlichkeit wird dabei verstanden, wie die Funktionalitäten der Anwendung dem Benutzer präsentiert werden. Die Herausforderung dabei ist, die Bedürfnisse der Benutzer zu verstehen und diese in die Software einfließen zu lassen \cite{usability}.\newline 
Eine verbesserte Benutzerfreundlichkeit ist dabei ausschlaggebend für gute Software. Sie wirkt sich auf die Effizienz und Produktivität, die Sicherheit, den Bedarf an Schulungen und die Akzeptanz der Software aus \cite{usability}.
Gemäß \cite{usability} wird die Benutzerfreundlichkeit mit nachfolgenden Kriterien bewertet:
\begin{itemize}
	\item Dauer die zum Erlernen der Software notwendig ist
	\item Vorhersagbarkeit
	\item Synthetisierbarkeit: Möglichkeit der Beurteilung der letzten Operation
	\item Vertrautheit
	\item Generalisierbarbeit
	\item Konsistenz
	\item Dauer für die Erfüllung einer Aufgabe
	\item Fehlerrate von Benutzern
	\item Persistenz der Schnittstellenkonzepte
	\item Subjektive Zufriedenheit
	\item Flexibilität: Dialog-Initiative, Multithreading, Substituierbarkeit, Anpassbarkeit
	\item Robustheit: Beobachtbarkeit, Wiederherstellbarkeit, Reaktionsfähigkeit
\end{itemize}

\subsubsection{Norm für Gebrauchstauglichkeit - DIN EN ISO 9241-11}
Nachfolgender Text ist ein Auszug von \cite{DIN1} und legt eine Reihe von Aktivitäten fest, die zur Erreichung der Gebrauchstauglichkeit angewendet werden können: \newline
\newline
a) Verstehen und Beschreiben des Nutzungskontextes:\newline
Diese Aktivität legt fest, was der/die maßgebende(n) Nutzungskontext(e) ist/sind, so dass die Informationen in der Anforderungsspezifikation verwendet werden können, sowie bei der Überlegung, welche Gestaltungslösungen in diesen Kontexten nutzbar sind.\newline
Gebrauchstauglichkeit kann für verschiedene Anwendungsbereiche eines Nutzungskontextes betrachtet werden, zum Beispiel: Alle potenziell maßgebenden Nutzungskontexte, Ausgewählte Nutzungskontexte, Ein Einzelfall des Nutzungskontextes, Nutzungskontext für eine Einzelperson.\newline
\newline
b) Spezifizierung der Benutzeranforderungen:\newline
Die Identifizierung von Benutzerbedürfnissen kann die Grundlage für die Spezifizierung von Benutzeranforderungen bilden. Die Spezifizierung der Benutzeranforderungen umfasst die Identifizierung der maßgeblichen Anforderungen an die Gebrauchstauglichkeit (d. h. Kriterien für Effektivität, Effizienz und Zufriedenstellung in bestimmten Nutzungskontexten). \newline
Beispiel: Für eine E-Commerce-Website könnten die wichtigsten Ziele Effektivität (Genauigkeit und Vollständigkeit) und Zufriedenstellung (positive Einstellungen und Emotionen einschließlich Vertrauen und Spaß) für einen gelegentlichen Benutzer sein.\newline
\newline
c) Gestaltungslösungen entwerfen:\newline
Diese Aktivität beinhaltet u. a. die Anwendung des Nutzungskontextes und der Benutzeranforderungen in Verbindung mit ergonomischen Kenntnissen, um zum Entwurf einer gebrauchstauglichen Gestaltungslösung beizutragen.\newline
\newline
d) Evaluierung der Gestaltung:\newline
Diese Aktivität evaluiert mögliche Gestaltungslösungen und/oder bereits vorhandene Produkte oder Dienstleistungen und bietet Rückmeldungen zu ihrer Gebrauchstauglichkeit.\newline
Die spezifischen Ziele, für die die Gebrauchstauglichkeit des interessierenden Objekts betrachtet wird, können zur Identifizierung der jeweiligen Bedeutung von Effektivität, Effizienz und Zufriedenstellung oder ihrer Elemente sowie zu einem oder mehreren Maßen für jedes dieser Merkmale führen. Es ist wichtig, dass diese Maße sich glaubwürdig und verlässlich auf die identifizierten Ziele beziehen.\newline
Die Evaluierung kann sich auf das/die Endergebnis(se) oder auf die Ergebnisse von Interaktionen unter Nutzung des interessierenden Objektes konzentrieren.\newline
Gestaltungslösungen oder Systeme können evaluiert werden, um festzustellen, ob sie die Benutzer-anforderungen innerhalb eines bestimmten Nutzungskontextes erfüllen, und/oder um Probleme im Zusammenhang mit der Gebrauchstauglichkeit innerhalb eines bestimmten Nutzungskontextes zu erkennen.

\subsubsection{Norm für Dialoggestaltung - DIN EN ISO 9241-110}
Nachfolgender Text stammt von \cite{DIN2} und stellt Grundsätze vor, welche bei der Dialoggestaltung berücksichtigt werden sollten: \newline
\\
Die folgenden sieben Grundsätze sind für die Gestaltung und Bewertung eines Dialoges wichtig:\newline
\newline
a) Aufgabenangemessenheit:\newline
Ein interaktives System ist aufgaben-angemessen, wenn es den Benutzer unterstützt, seine Arbeitsaufgabe zu
erledigen, d. h., wenn Funktionalität und Dialog auf den charakteristischen Eigenschaften der Arbeitsaufgabe
basieren, anstatt auf der zur Aufgabenerledigung eingesetzten Technologie. Darunter fällt:
\begin{itemize}
	\item Der Dialog sollte dem Benutzer solche Informationen anzeigen, die im Zusammenhang mit der
	erfolgreichen Erledigung der Arbeitsaufgabe stehen.
	\item Der Dialog sollte dem Benutzer keine Informationen anzeigen, die nicht für die erfolgreiche Erledigung relevanter Arbeitsaufgaben benötigt werden.
	\item Die Form der Eingabe und Ausgabe sollte der Arbeitsaufgabe angepasst sein.
	\item Wenn für eine Arbeitsaufgabe ganz bestimmte Eingabewerte typisch sind, sollten diese Werte dem
	Benutzer automatisch als voreingestellte Werte verfügbar sein.
	\item Die vom interaktiven System verlangten Dialogschritte sollten zum Arbeitsablauf passen, d. h.,
	notwendige Dialogschritte sollten enthalten sein und unnötige Dialogschritte sollten vermieden werden.
	\item Wenn bei einer Arbeitsaufgabe Quelldokumente verwendet werden, sollte die Benutzungsschnittstelle
	kompatibel zu den charakteristischen Eigenschaften der Quelldokumente sein.
	\item Die Eingabe- und Ausgabemedien des interaktiven Systems sollten aufgabenangemessen sein.
\end{itemize}
b) Selbstbeschreibungsfähigkeit:\newline
Ein Dialog ist in dem Maße selbstbeschreibungsfähig, in dem für den Benutzer zu jeder Zeit offensichtlich ist,
in welchem Dialog, an welcher Stelle im Dialog er sich befindet, welche Handlungen unternommen werden
können und wie diese ausgeführt werden können. Hierfür gilt:
\begin{itemize}
	\item Die bei jedem Dialogschritt angezeigten Informationen sollten den Benutzer leiten, den Dialog
	erfolgreich abzuschließen.
	\item Während der Interaktion mit dem System sollte die Notwendigkeit, Benutzer-Handbücher und andere
	externe Informationen heranzuziehen, minimiert sein.
	\item Der Benutzer sollte über Änderungen des Zustandes des interaktiven Systems informiert werden, also wann Eingaben erwartet werden und durch Bereitstellung eines Überblickes über die nächsten Dialogschritte.
	\item Wenn eine Eingabe verlangt wird, sollte das interaktive System dem Benutzer Informationen über die
	erwartete Eingabe bereitstellen.
	\item Dialoge sollten so gestaltet sein, dass die Interaktion für den Benutzer offensichtlich ist.
	\item Das interaktive System sollte dem Benutzer Informationen über die erforderlichen Formate und
	Einheiten bereitstellen.
\end{itemize}
c) Erwartungskonformität:\newline
Ein Dialog ist erwartungskonform, wenn er den aus dem Nutzungskontext heraus vorhersehbaren Benutzerbelangen
sowie allgemein anerkannten Konventionen entspricht.
\begin{itemize}
	\item Das interaktive System sollte das Vokabular verwenden, das dem Benutzer bei der Ausführung der
	Arbeitsaufgabe vertraut ist oder von ihm auf Grund seiner Kenntnisse und Erfahrungen verwendet wird.
	\item Auf Handlungen des Benutzers sollte eine unmittelbare und passende Rückmeldung folgen, soweit
	dies den Erwartungen des Benutzers entspricht.
	\item Kann vorhergesehen werden, dass erhebliche Abweichungen von der vom Benutzer erwarteten
	Antwortzeit entstehen, sollte der Benutzer hiervon unterrichtet werden.
	\item Informationen sollten so strukturiert und organisiert sein, wie es vom Benutzer als natürlich
	empfunden wird.
	\item Formate sollten geeigneten kulturellen und sprachlichen Konventionen entsprechen.
	\item Art und Länge von Rückmeldungen oder Erläuterungen sollten den Benutzerbelangen entsprechen.
	\item Dialogverhalten und Informationsdarstellung eines interaktiven Systems sollten innerhalb von
	Arbeitsaufgaben und über ähnliche Arbeitsaufgaben hinweg konsistent sein.
	\item Wenn eine bestimmte Eingabeposition auf der Grundlage von Benutzererwartungen vorhersehbar ist,
	dann sollte diese Position für die Eingaben voreingestellt sein.
	\item Rückmeldungen oder Mitteilungen, die dem Benutzer angezeigt werden, sollten in einer objektiven
	und konstruktiven Art formuliert sein.
\end{itemize}
d) Lernförderlichkeit:\newline
Ein Dialog ist lernförderlich, wenn er den Benutzer beim Erlernen der Nutzung des interaktiven Systems
unterstützt und anleitet.
\begin{itemize}
	\item Regeln und zugrunde liegende Konzepte, die für das Erlernen nützlich sind, sollten dem Benutzer
	zugänglich gemacht werden.
	\item Wenn ein Dialog selten gebraucht wird oder charakteristische Eigenschaften des Benutzers es
	erfordern, den Dialog erneut zu erlernen, dann sollte geeignete Unterstützung dafür bereitgestellt werden.
	\item Geeignete Unterstützung sollte bereitgestellt werden, damit der Benutzer mit dem Dialog vertraut wird.
	\item Rückmeldung und Erläuterungen sollten den Benutzer unterstützen, ein konzeptionelles Verständnis
	vom interaktiven System zu bilden.
	\item Der Dialog sollte ausreichende Rückmeldung über Zwischen- und Endergebnisse von Handlungen
	bereitstellen, damit die Benutzer von erfolgreich ausgeführten Handlungen lernen.
	\item Falls es zu den Arbeitsaufgaben und den Lernzielen passt, sollte das interaktive System dem
	Benutzer erlauben, Dialogschritte ohne nachteilige Auswirkungen neu auszuprobieren.
	\item Das interaktive System sollte es dem Benutzer ermöglichen, die Arbeitsaufgabe mit minimalem
	Lernaufwand auszuführen, indem es den Dialog mit minimaler Eingabe von Informationen ermöglicht, jedoch
	zusätzliche Information auf Anforderung zur Verfügung stellt.
\end{itemize}
e) Steuerbarkeit:\newline
Ein Dialog ist steuerbar, wenn der Benutzer in der Lage ist, den Dialogablauf zu starten sowie seine Richtung
und Geschwindigkeit zu beeinflussen, bis das Ziel erreicht ist.
\begin{itemize}
	\item Die Geschwindigkeit der Interaktion sollte nicht durch das interaktive System vorgegeben werden. Sie
	sollte vom Benutzer steuerbar sein, und zwar unter Berücksichtigung der Benutzerbelange und der
	charakteristischen Eigenschaften des Benutzers.
	\item Der Benutzer sollte die Steuerung darüber haben, wie der Dialog fortgesetzt wird.
	\item Ist der Dialog unterbrochen worden, sollte der Benutzer die Möglichkeit haben, den
	Wiederaufnahmepunkt der Fortsetzung des Dialoges zu bestimmen, falls es die Arbeitsaufgabe erlaubt.
	\item Wenigstens der letzte Dialogschritt sollte zurückgenommen werden können, soweit Handlungsschritte
	reversibel sind und falls es der Nutzungskontext erfordert.
	\item Wenn die Datenmenge, die für eine Arbeitsaufgabe von Bedeutung ist, groß ist, dann sollte der
	Benutzer die Möglichkeit haben, die Anzeige der dargestellten Datenmenge zu steuern.
	\item Der Benutzer sollte dort, wo es geeignet ist, die Möglichkeit haben, jedes verfügbare 	Eingabe-/Ausgabemittel benutzen zu können.
	\item Wenn es für die Arbeitsaufgabe zweckmäßig ist, sollte der Benutzer voreingestellte Werte ändern
	können.
	\item Wenn Daten verändert wurden, sollten die Originaldaten für den Benutzer verfügbar bleiben, wenn
	dies für die Arbeitsaufgabe erforderlich ist.
\end{itemize}
f) Fehlertoleranz:\newline
Ein Dialog ist fehlertolerant, wenn das beabsichtigte Arbeitsergebnis trotz erkennbar fehlerhafter Eingaben
entweder mit keinem oder mit minimalem Korrekturaufwand seitens des Benutzers erreicht werden kann.
Fehlertoleranz wird erreicht mit Fehlererkennung und -vermeidung (Schadensbegrenzung), Fehlerkorrektur oder Fehlermanagement, um mit Fehlern umzugehen, die sich ereignen.
\begin{itemize}
	\item Das interaktive System sollte den Benutzer dabei unterstützen, Eingabefehler zu entdecken und zu
	vermeiden.
	\item Das interaktive System sollte verhindern, dass irgendeine Benutzer-Handlung zu undefinierten
	Systemzuständen oder zu Systemabbrüchen führen kann.
	\item Wenn sich ein Fehler ereignet, sollte dem Benutzer eine Erläuterung zur Verfügung gestellt werden,
	um die Beseitigung des Fehlers zu erleichtern.
	\item Aktive Unterstützung zur Fehlerbeseitigung sollte dort, wo typischerweise Fehler auftreten, zur
	Verfügung stehen.
	\item Wenn das interaktive System Fehler automatisch korrigieren kann, sollte es den Benutzer über die
	Ausführung der Korrektur informieren und ihm Gelegenheit geben, zu korrigieren.
	\item Der Benutzer sollte die Möglichkeit haben, die Fehlerkorrektur zurückzustellen oder den Fehler
	unkorrigiert zu lassen, es sei denn, eine Korrektur ist erforderlich, um den Dialog fortsetzen zu können.
	\item Wenn möglich, sollten dem Benutzer auf Anfrage zusätzliche Informationen zum Fehler und dessen
	Beseitigung zur Verfügung gestellt werden.
	\item Die Prüfung auf Gültigkeit und Korrektheit von Daten sollte stattfinden, bevor das interaktive System
	die Eingabe verarbeitet.
	\item Die zur Fehlerbehebung erforderlichen Schritte sollten minimiert sein.
	\item Falls sich aus einer Benutzerhandlung schwerwiegende Auswirkungen ergeben können, sollte das
	interaktive System Erläuterungen bereitstellen und Bestätigung anfordern, bevor die Handlung ausgeführt
	wird.
\end{itemize}
g) Individualisierbarkeit:\newline
Ein Dialog ist individualisierbar, wenn Benutzer die Mensch-System-Interaktion und die Darstellung von
Informationen ändern können, um diese an ihre individuellen Fähigkeiten und Bedürfnisse anzupassen.
\begin{itemize}
	\item Das interaktive System sollte dem Benutzer dort, wo unterschiedliche Benutzerbelange typischerweise
	vorkommen, Techniken zur Anpassung an die charakteristischen Eigenschaften von Benutzern
	bereitstellen.
	\item Das interaktive System sollte es dem Benutzer erlauben, zwischen verschiedenen Formen der
	Darstellung zu wählen, wenn es für die individuellen Bedürfnisse unterschiedlicher Benutzer zweckmäßig ist.
	\item Der Umfang von Erläuterungen (z. B. Details in Fehlermeldungen, Hilfeinformationen) sollte
	entsprechend dem individuellen Wissen des Benutzers veränderbar sein.
	\item Benutzer sollten, soweit zweckmäßig, die Möglichkeit haben, eigenes Vokabular einzubinden, um
	Objekte und Funktionen („Werkzeuge“) individuell zu benennen.
	\item Der Benutzer sollte, soweit zweckmäßig, die Geschwindigkeit von dynamischen Eingaben und
	Ausgaben einstellen können, um sie an seine individuellen Bedürfnisse anzupassen.
	\item Die Benutzer sollten, soweit zweckmäßig, die Möglichkeit haben, zwischen unterschiedlichen
	Dialogtechniken zu wählen.
	\item Der Benutzer sollte die Möglichkeit haben, das Niveau und die Methoden der Mensch-System-
	Interaktion so auszuwählen, dass sie am besten seinen Bedürfnissen entsprechen.
	\item Der Benutzer sollte die Möglichkeit haben, die Art zu wählen, in der Eingabe-/Ausgabe-Daten
	dargestellt werden (Format und Typ).
	\item Soweit zweckmäßig, sollte es den Benutzern möglich sein, Dialogelemente oder Funktionen
	hinzuzufügen oder neu zu ordnen, insbesondere, um individuelle Bedürfnisse bei der Ausführung von
	Arbeitsaufgaben zu unterstützen.
	\item Individuelle Einstellungen eines Dialoges sollten rückgängig gemacht werden können und es dem
	Benutzer erlauben, zu den ursprünglichen Einstellungen zurückzugehen.
\end{itemize}
\newpage
